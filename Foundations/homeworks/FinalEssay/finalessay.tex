\documentclass[12pt]{article}
\usepackage[letterpaper]{geometry}
\usepackage{times}
\usepackage{rotating}
\usepackage{multirow}
\usepackage{fancyhdr}
\usepackage{amsmath,amssymb}
\usepackage{color}
\usepackage{graphicx}
\usepackage{hyperref}
%\usepackage{url}
\usepackage{fullpage}

\pagestyle{fancy}
\lhead{} 
\chead{} 
\rhead{Lee \thepage} 
\lfoot{} 
\cfoot{} 
\rfoot{} 
\renewcommand{\headrulewidth}{0pt} 
\renewcommand{\footrulewidth}{0pt} 
%To make sure we actually have header 0.5in away from top edge
%12pt is one-sixth of an inch. Subtract this from 0.5in to get headsep value
\setlength\headsep{0.333in}
\newcommand{\bibent}{\noindent \hangindent 40pt}
\newenvironment{workscited}{\newpage \begin{center} Works Cited \end{center}}{\newpage }

\geometry{top=1.0in, bottom=1.0in, left=1.0in, right=1.0in}

\usepackage{setspace}
\doublespacing
\begin{document}

\begin{flushleft}

Sunny Lee\\
Amanda Hattaway \\
Foundations of Applied Math \\
November 23 2020\\

\begin{center}
  Actuaries and Data Science during COVID-19
\end{center}

\setlength{\parindent}{0.5in} In light of COVID-19, people have had to adapt their 
lifestyles to ensure the safety of people and reduce the spread of COVID-19. These 
restrictions, however do not come from nowhere. Actuaries and data scientists 
analyze the data coming from those effected by COVID-19 in order to try and model
certain important aspects of daily life. Some aspects include hospital beds which 
could be at a premium in highly susceptible areas. In order to inform as many 
people as possible, actuaries may want to explain very complex models so more 
people can understand what certain models are predicting.

With COVID-19 being so prevalent in daily life, there is a great deal of data which 
actuaries and data scientists can use. Using data, actuaries can model certain aspects 
of COVID-19 related consequences such as hospital bed utilization and virus infection rate. 
By building these models, actuaries can extrapolate and try to predict what is coming 
in the near future. One such model could help with the utilization of beds in hospitals
and the virus infection rates of COVID-19\cite{Eaton}.
Modeling hospital bed utilization will help hospitals use their beds much more 
efficiently which will decrease congestion in hospitals and allow for more people 
who need care to get what they need. Modeling virus infection rates will also show 
how susceptible people are to the virus and how likely it is for the virus to spread 
in densely populated areas. This kind of statistical analysis can help 
give advice to the general population in order to control the spread and overall 
consequences COVID-19 has on the world. 

Actuaries are also very versed in communicating. Their job usually requires them to 
explain what they have done in layman's terms. This ability to translate complex 
models and put them in a way that anyone can understand is crucial to spreading the 
knowledge about COVID-19. Actuaries can also give insight into models and compare 
those models to other models\cite{Eaton}. Since there are many ways of looking 
at and analyzing data, it is always good to compare models to see what other people 
have done. Some models will make different assumptions which could result in 
drastically different looking models. Some models may even be made to misguide 
instead of inform, so actuaries could call these models out as false. 

COVID-19 has brought on a mandatory surge of data science and actuarial work. With 
massive amounts of data to use, there are many ways one could interpret that data. 
By using COVID-19 data, actuaries can effectively model hospital bed utilization 
to prevent hospitals from gettting flooded. However, these models are also 
very complex and might not be very intuitive. Thus, it is also important for 
actuaries to communicate about their models as well as others. 

\newpage
\begin{thebibliography}{98}

  \bibitem{Eaton} Eaton, Robert. 2020. Actuaries in the Time of Coronavirus. \href{https://theactuarymagazine.org/actuaries-in-the-time-of-coronavirus/}{https://theactuarymagazine.org/actuaries-in-the-time-of-coronavirus/}. Accessed 23 November 2020.

\end{thebibliography}

\end{flushleft}
\end{document}
\}